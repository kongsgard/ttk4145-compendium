\ctitle{Scheduling Real-time Systems}

\paragraph{This chapter} is based on Ch. 11 in (BW).

\sepline

\paragraph{Scheduling}\index{scheduling} Restricting the non-determinism found within concurrent systems, by providing
\begin{itemize}[nolistsep,noitemsep]
  \item An algorithm for ordering the use of system resources.
  \item A means of predicting the worst case behaviour under this ordering.
\end{itemize}
Ie. it has to assign priorities to the different processes, and run a schedulability test. In a \textit{static} scheduler this is decided/calculated prior to execution, while a \textit{dynamic} scheduler calculates in run-time.

The scheduler must be predictable, so that we can analyze the system. We want the analysis to be simple, but not too conservative, as we want to facilitate high utilization. The result of the analysis must show that all tasks meet their deadlines.

\paragraph{Task-based scheduling}\index{scheduling!task-based} An approach where task execution is directly supported. A task is deemed to be in a specific \textit{state}:
\begin{itemize}[nolistsep,noitemsep]
  \item Runnable
  \item Suspended waiting for a timing event -- appropriate for periodic tasks
  \item Suspended waiting for a non-timing event -- appropriate for sporadic tasks
\end{itemize}
We consider three scheduling approaches:
\begin{itemize}[nolistsep,noitemsep]
  \item \textit{Fixed-Priority Scheduling (FPS)}. Each task has a fixed, static, priority which is computed pre-run-time. This is the most widely used approach.
  \item \textit{Earliest Deadline First (EDF) Scheduling}. The tasks are executed in the order determined by the absolute deadlines (earliest deadline yields highest priority). Typically computed at run-time, so this scheme is dynamic.
  \item \textit{Value-Based Scheduling (VBS)}. Assigning a value to each task to decide which to run next. Adaptive.
\end{itemize}

\paragraph{Scheduling characteristics}\index{scheduling!characteristics} A schedulability test can be:
\begin{itemize}[nolistsep,noitemsep]
  \item \textit{Sufficient}.
  \item \textit{Necessary}.
  \item \textit{Sustainable}. The test still predicts schedulability when the system parameters (eg. period, deadline) improve.
\end{itemize}

\paragraph{Preemption}\index{preemption} In a preemptive scheme, the scheduler automatically switches to tasks of higher priority.

(In non-preemptive schemes, the lower-priority task will be allowed to complete before the other executes.)

\paragraph{Simple task model} In the following discussion we make the following assumptions of the tasks (comments on how realistic they are in paranthesis):
\begin{itemize}[nolistsep,noitemsep]
  \item Fixed set of tasks (No sporadic tasks. Not optimal, but can be worked around)
  \item Periodic tasks with known periods (Realistic in many systems)
  \item The tasks are independent (Completely realistic in an embedded system)
  \item Overheads, switching times can be ignored (Depends)
  \item Deadline == Period (Inflexible, but fair enough)
  \item Fixed worst-case execution time (Not realistic to know a tight (not overly conservative) estimate here)
  \item Rate-monotonic priority ordering (Our choice, so ok)
\end{itemize}

\hskip-0.5cm
\begin{tabularx}{\linewidth}{c X}
	\multicolumn{2}{c}{\textbf{Standard notation}} \\
	Notation & Description \\
	\hline
	$B$ & Worst-case blocking time for the task (if applicable)\\
  $C$ & Worst-case execution time (WCET) of the task\\
  $D$ & Deadline of the task\\
  $I$ & The interference time of the task\\
  $J$ & Release jitter of the task\\
  $N$ & Number of tasks in the system\\
  $P$ & Priority assigned to the task (if applicable)\\
  $R$ & Worst-case response time of the task\\
  $T$ & Minimum time between task releases (task period)\\
  $U$ & The utilization of each task (equal to $C/T$)\\
  $a$ -- $z$ & The name of a task
\end{tabularx}

\paragraph{Fixed-priority scheduling (FPS)} ~\\
\textit{Rate monotonic} priority assignment: shortest period yields highest priority. (For two tasks $i$ and $j$, $T_i < T_j \Rightarrow P_i > P_j$). The higher value of $P$, the greater the priority (the other way around compared to other disciplines).

\textit{Utilization test}. If the following condition is true then all $N$ tasks will meet their deadlines:
\begin{equation}
  U = \sum_{i=1}^N \left( \frac{C_i}{T_i} \right) \leq N\left( 2^{1/N} - 1 \right)
\end{equation}
The test is sufficient, but not necessary (a system may fail the test, and still be schedulable). Additionally, two drawbacks: not exact, not applicable to a more general task model.

\textit{Response time analysis (RTA)}. Exact (sufficient and necessary) test.
\begin{equation}
  w_i^{n+1} = C_i + \sum_{j \in hp(i)} \ceil*{\frac{w_i^n}{T_j}} C_j
\end{equation}
\begin{align*}
  w_i &= \text{A mathematical entity to solve the fixed-point}\\ &\text{equation above, equal to $R_i$ after the final recursion.}\\
  hp(i) &= \text{The set of indices that have a higher priority than $i$}
\end{align*}
Recursive test. Start with the highest-priority task (which will have response time equal to its computation time), and move to the lower priority tasks. The response time of a task $i$ has been found if $w_i^{n+1} = w_i^n$.

\paragraph{Priority inversion}\index{priority inversion} That a high-priority task ends up waiting for a lower-priority task, which can happen if they share a resource. (In the previous discussion tasks have beeen viewed as independent, but tasks may share resources, and therefore have the possibility to be suspended; waiting for a future event (eg. an unlocked semaphore)).

\textit{Unbounded priority inversion}\index{priority inversion!unbounded} That the high-priority task may potentially end up waiting forever. The unboundedness occurs when there are one or more medium-priority tasks that prevent the low-priority task from running - and thereby releasing the shared resource.

\paragraph{Priority inheritance}\index{priority inheritance} If a task $p$ is suspended waiting for task $q$ to undertake som computation then the priority of $q$ becomes equal to the priority of $p$ (if it was lower to start with). Does NOT prevent deadlocks.

\paragraph{Priority ceiling}\index{priority ceiling} When this kind of protocol is used...
\begin{itemize}
  \item A high-priority task can be blocked at most once during its execution by lower-priority tasks
  \item Deadlocks are prevented
  \item Transitive blocking is prevented
  \item Mutual exclusive access to resources is ensured
\end{itemize}

The protocol ensures that if a resources is locked, by task $a$ say, and could lead to the blocking of a higher-priority task $b$, then no other resources that could block $b$ is allowed to be locked by any task other than $a$. A task can therefore be delayed by not only attempting to lock a previously locked resource, but also when the lock could lead to multiple blocking on higher-priority tasks.

\textit{Original ceiling priority protocol (OCPP)}:
\begin{itemize}
  \item Each task has a static default priority.
  \item Each resource has a static ceiling value (the maximum priority of the tasks that use it).
  \item Tasks have dynamic priority (its own static priority or any (higher) priorities inherited from blocking higher-priority tasks).
  \item A resource may only be locked by tasks with higher priority than any currently locked tasks.
\end{itemize}

\textit{Immediate ceiling priority protocol (ICPP)}:
% The same three first points as for OCPP?
Instead of waiting until it actually blocks a higher priority task, ICPP increases the priority of a task as soon as in locks a resource. ICPP will not start executing until all its needed resources are free, since unavailable resouces it needs indicates that a task of higher priority is running.

Ensures mutual exclusion: If a task under (eg.) ICPP has access to a resource it will have the highest possible priority for that resource, and thus no other task demanding this resource will run.

Ensures no deadlocks: If a task has a resource and requests another, then the new resource it requested can not have lower priority.

% TODO: cyclic executive approach, improved utilization tests, sporadic and aperiodic tasks
% 11.10 -> (not important)
