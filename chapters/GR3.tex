\ctitle{Fault Model and Software Fault Masking}

This chapter is based on Ch. 3.7 in the book Transaction Processing: Concepts and Techniques (GR).

\sepline

\paragraph{Fault model} The one used in this chapter involves three entities: \textit{processes (feil sideeffekt), messages and storage (static redundancy)}.

\paragraph{Unexpected faults}\index{unexpected faults} Faults that are not tolerated by the design. Two categorizations:
\begin{itemize}[nolistsep,noitemsep]
  \item \textit{Dense faults}: The algorithms will be \textit{n}-fault tolerant. If there are more than \textit{n} faults within a repair period, the service may be interrupted (in this case the system should be designed to stop).
  \item \textit{Byzantine faults}: The fault model postulates certain behavior -- for example it may postulate that programs are failfast. Faults in which the system does not conform to the model behavior are called Byzantine.
\end{itemize}

\paragraph{Underlying progression}\index{process}
\begin{itemize}
  \item \textit{Failfast}: They either execute the next step, or they fail and reset to the null state.
  \item \textit{Available}: Failfast + repairability
  \item \textit{Reliable}: Continuous operation
\end{itemize}

\paragraph{Checkpoint-Restart}\index{Checkpoint-Restart} Write state to storage after each acceptance test, but before each event. This approach can be somewhat slow (hours/days).

\paragraph{Process pairs}\index{process pairs} OS generates a backup process for each new primary process. The primary sends \textit{I'm Alive} messages to the backup on a regular basis.

The backup can take over in three different ways:
\begin{itemize}
  \item \textit{Checkpoint-restart.} The primary records its state on a duplexed storage module. At takeover, the backup starts by reading these duplexed storage pages. (quick repair)
  \item \textit{Checkpoint message.} The primary sends its state changes as messages to the backup. At takeover, the backup gets its current state from the most recent message. (basic pairs must checkpoint)
  \item \textit{Persistent.} The backup restarts in the null state and lets the transaction mechanism clean up (undo) any recent uncommitted state changes. (simple)
\end{itemize}

\tikzstyle{startstop} = [rectangle, rounded corners, minimum width=3cm, minimum height=0.5cm,text centered, draw=black, fill=red!30]
\tikzstyle{process} = [rectangle, minimum width=4cm, minimum height=0.5cm, text centered, text width=5cm, draw=black, fill=orange!30]
\tikzstyle{decision} = [diamond, aspect=2, minimum width=3cm, minimum height=1cm, text centered, text width=2cm, draw=black, fill=green!30]
\tikzstyle{farrow} = [single arrow, minimum height=2cm, draw=black]

\tikzstyle{arrow} = [thick,->,>=stealth]

% --- %

\resizebox{10cm}{!} {%
\begin{tikzpicture}[align=center, node distance=0.8cm]

\node (start) [startstop] {Restart};
\node (dec1) [decision, below of=start, yshift=-0.7cm] {Am I default primary?};
\node (pro1a) [process, below of=dec1, yshift=-0.7cm] {Wait a second};
\node (dec2) [decision, below of=pro1a, yshift=-0.5cm] {Any input?};
\node (pro2) [process, below of=dec2, yshift=-0.5cm] {Read it};
\node (dec3) [decision, below of=pro2, yshift=-0.5cm] {Newer state?};
\node (pro3) [process, below of=dec3, yshift=-0.5cm] {Set my state to new state};

\node (pro1b) [process, left of=dec1, xshift=-5cm] {Wait a second};
\node (dec4) at (pro1b|-dec2) [decision] {New state in last second?};

\node (pro4) [process, left of=dec4, xshift=-5cm] {Broadcast: I'm Primary};
\node (pro5) [process, below of=pro4] {Reply to last request};
\node (dec5) [decision, below of=pro5, yshift=-0.5cm] {Any input?};
\node (pro6) [process, below of=dec5, yshift=-0.5cm] {Read it};
\node (pro7) [process, below of=pro6] {Compute new state};
\node (pro8) [process, below of=pro7] {Send new state to backup};
\node (pro9) [process, below of=pro8] {Reply};

\node (pro10) at (pro1b|-pro8)[process] {Send state to backup};

\node [farrow, below of=pro3, yshift=-0.25cm] {I'm alive};
\node [farrow, left of=pro6, xshift=-3.4cm] {Requests};
\node [farrow, left of=pro9, xshift=-3.4cm] {Replies};

% --- %

\draw [arrow] (start) -- (dec1);
\draw [arrow] (dec1) -- (pro1a);
\draw [arrow] (pro1a) -- (dec2);
\draw [arrow] (dec2) -- node[anchor=west, yshift=0.05cm] {+} (pro2);
\draw [arrow] (pro2) -- (dec3);
\draw [arrow] (dec3) -- node[anchor=west, yshift=0.05cm] {+} (pro3);
\draw [arrow] (pro3) --([xshift={3mm}]pro3.east)|- (dec2);
\draw [arrow] (dec3) --([xshift={3mm}]pro3.east)|- (dec2);

\draw [arrow] (dec1) -- node[anchor=north] {--} (pro1b);
\draw [arrow] (dec2) -- node[anchor=north] {--} (dec4);

\draw [arrow] (dec4) -- (pro4);

\draw [arrow] (pro4) -- (pro5);
\draw [arrow] (pro5) -- (dec5);
\draw [arrow] (dec5) -- node[anchor=west, yshift=0.05cm] {+} (pro6);
\draw [arrow] (pro6) -- (pro7);
\draw [arrow] (pro7) -- (pro8);
\draw [arrow] (pro8) -- (pro9);

\draw [arrow] (dec5) -| (pro10);
\draw [arrow] (pro10) |-([shift={(-3mm,-3mm)}]pro9.south west)|- (dec5);
\draw [arrow] (pro9) |-([shift={(-3mm,-3mm)}]pro9.south west)|- (dec5);

\end{tikzpicture}
}


% TODO: Transactions, persistent processes, acceptancy test
