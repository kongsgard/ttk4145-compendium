\newpage
\ctitle{Code Quality}

%\paragraph{This chapter} ...
%\sepline

\begin{multicols}{2}
%%%
\section*{CHECKLIST:\\Class Quality}
%%%

\subsection*{Abstract Data Types}
%
\begin{todolist}
  \item Have you thought of the classes in your program as abstract data types
and evaluated their interfaces from that point of view?
\end{todolist}

\subsection*{Abstraction}
\begin{todolist}
  \item Does the class have a central purpose?
  \item Is the class well named, and does its name describe its central purpose?
  \item Does the class's interface present a consisten abstraction?
  \item Does the class's interface make obvious how you should use the class?
  \item Is the class's interface abstract enough that you don't have to think about how its services are implemented? Can you treat the class as a black box?
  \item Are the class's services complete enough that other classes don't have to meddle with its internal data?
  \item Has unrelated information been moved out of the class?
  \item Have you thought about subdividing the class into component class, and have you subdivided it as much as you can?
  \item Are you preserving the integrity of the class's interface as you modify the class?
\end{todolist}

\subsection*{Encapsulation}
\begin{todolist}
  \item Does the class minimize accessibility to its members?
  \item Does the class avoid exposing member data?
  \item Does the class hide its implementation details from other classes as much as the programming language permits?
  \item Does the class avoid making assumptions about its users, including its derived classes?
  \item Is the class independent of other classes? Is it loosely coupled?
\end{todolist}

\subsection*{Inheritance}
\begin{todolist}
  \item Is inheritance used only to model "is a" relationships -- that is, do derived classes adhere to the Liskov Substitution Principle?
  \item Does the class documentation describe inheritance strategy?
  \item Do derived classes avoid "overriding" non-overridable routines?
  \item Are common interfaces, data, and behavior as high as possible in the inheritance tree?
  \item Are inheritance trees fairly shallow?
  \item Are all data members in the base class private rather than protected?
\end{todolist}

\subsection*{Other Implementation Issues}
\begin{todolist}
  \item Does the class contain about seven data members or fewer?
  \item Does the class minimize direct and indirect routine calls to other classes?
  \item Does the class collaborate with other classes only to the extent absolutely necessary?
  \item Is all member data initialized in the constructor?
  \item Is the class designed to be used as deep copies rather than shallow copies unless there's a measured reason to create shallow copies?
\end{todolist}

\subsection*{Language-Specific Issues}
\begin{todolist}
  \item Have you investigated the language-specific issues for classes in your specific programming language?
\end{todolist}

%%%
\section*{CHECKLIST:\\High-Quality Routines}
%%%

\subsection*{Big-Picture Issues}
\begin{todolist}
  \item Is the reason for creating the routine sufficient?
  \item Have all parts of the routine that would benefit from being put into routines of their own been put into routines of their own?
  \item Is the routine's name a strong, clear verb-plus-object name for a procedure or a description of the return value for a function?
  \item Does the routine's name describe everything the routine does?
  \item Have you established naming conventions for common operations?
  \item Does the routine have strong, functional cohesion -- doing one and only one thing and doing it well?
  \item Do the routines a loose coupling -- are the routine's connections to other routines small, intimate, visible, and flexible?
  \item Is the length of routine determined naturally by its function and logic, rather than by an artificial coding standard?
\end{todolist}

\subsection*{Parameter-Passing Issues}
\begin{todolist}
  \item Does the routine's parameter list, taken as a whole, present a consistent interface abstraction?
  \item Are the routine's parameters in a sensible order, including matching the order of parameters in similar routines?
  \item Are interface assumptions documented?
  \item Does the routine have seven or fewer parameters?
  \item Is each input parameter used?
  \item Is each output parameter used?
  \item Does the routine avoid using input parameters as working variables?
  \item If the routine is a function, does it return a valid value under all possible circumstances?
\end{todolist}

%%%
\section*{CHECKLIST:\\Naming Variables}
%%%

\subsection*{General Naming Considerations}
\begin{todolist}
  \item Does the name fully and accurately describe what the variable represents?
  \item Does the name refer to the real-worls problem rather than to the programming-language solution?
  \item Is the name long enough that you don't have to puzzle it out?
  \item Are computed-value qualifiers, if any, at the end of the name?
  \item Does the name use \textit{Count} or \textit{Index} instead of \textit{Num}?
\end{todolist}

\subsection*{Naming Specific Kinds of Data}
\begin{todolist}
  \item Are loop index names meaningful (something other than \textit{i}, \textit{j}, or \textit{k} if the loop is more than one or two lines long or is nested)?
  \item Have all "temporary" variables been renamed to something more meaningful?
  \item Are boolean variables named so that their meanings when they're \textit{true} are clear?
  \item Do enumerated-type names include a prefix or suffix that indicates the category -- for example, \textit{Color\_} for \textit{Color\_Red}, \textit{Color\_Green}, \textit{Color\_Blue}, and so on?
  \item Are named constants named for the abstract entities they represent rather than the numbers they refer to?
\end{todolist}

\subsection*{Naming Conventions}
\begin{todolist}
  \item Does the convention distinguish among local, class, and global data?
  \item Does the convention distinguish among type names, named constants, enumerated types, and variables?
  \item Does the convention identify input-only parameters to routines in languages that don't enforce them?
  \item Is the convention as compatible as possible with standard conventions for the language?
  \item Are names formatted for readability?
\end{todolist}

\subsection*{Short Names}
\begin{todolist}
  \item Does the code use long names (unless it's necessary to use short ones)?
  \item Does the code avoid abbreviations that save only one character?
  \item Are all words abbreviated consistently?
  \item Are the names pronounceable?
  \item Are names that could be misread or mispronounced avoided?
  \item Are short names documented in translation tables?
\end{todolist}

\subsection*{Common Naming Problems: Have You Avoided...}
\begin{todolist}
  \item ... names that are misleading?
  \item ... names with similar meanings?
  \item ... names that are different by only one or two characters?
  \item ... names that sound similar?
  \item ... names that use numerals?
  \item ... names intentionally misspelled to make them shorter?
  \item ... names that are commonly misspelled in English?
  \item ... names that conflict with standard library routine names or with predefined variable names?
  \item ... totally arbitrary names?
  \item ... hard-to-read characters?
\end{todolist}

%%%
\section*{CHECKLIST:\\Self-Documenting Code}
%%%

\subsection*{Classes}
\begin{todolist}
  \item Does the class's interface present a consistent abstraction?
  \item Is the class well named, and does its name describe its central purpose?
  \item Does the class's interface make obvious how you should use the class?
  \item Is the class's interface abstract enough that you don't have to think about how its services are implemented? Can you treat the class as a black box?
\end{todolist}

\subsection*{Routines}
\begin{todolist}
  \item Does each routine's name describe exactly what the routine does?
  \item Does each routine perform one well-defined task?
  \item Have all parts of each routine that would benefit from being put into their own routines been put into their own routines?
  \item Is each routine's interface obvious and clear?
\end{todolist}

\subsection*{Data Names}
\begin{todolist}
  \item Are type names descriptive enough to help document data declarations?
  \item Are variables named well?
  \item Are variables used only for the purpose for which they're named?
  \item Are loop counters given more informative names than \textit{i}, \textit{j}, and \textit{k}?
  \item Are well-named enumerated types used instead of makeshift flags or boolean variables?
  \item Are named constants used instead of magic numbers or magic strings?
  \item Do naming conventions distinguis among type names, enumerated types, named constants, local variables, class variables, and global variables?
\end{todolist}

\subsection*{Data Organization}
\begin{todolist}
  \item Are extra variables used for clarity when needed?
  \item Are references to variables close together?
  \item Are data types simple so that they minimize complexity?
  \item Is complicated data accessed through abstract access routines (abstract data types)?
\end{todolist}

\subsection*{Control}
\begin{todolist}
  \item Is the nominal path through the code clear?
  \item Are related statements grouped together?
  \item Have relatively independent groups of statements been packaged into their own routines?
  \item Does the normal case follow the \textit{if} rather than the \textit{else}?
  \item Are control structures simple so that they minimize complexity?
  \item Does each loop perform one and only one function, as a well-defined routine would?
  \item Is nesting minimized?
  \item Have boolean expressions been simplified by using additional boolean variables, boolean functions, and decision tables?
\end{todolist}

\subsection*{Layout}
\begin{todolist}
  \item Does the program's layout show its logical structure?
\end{todolist}

\subsection*{Design}
\begin{todolist}
  \item Is the code straightforward, and does it avoid cleverness?
  \item Are implementation details hidden as much as possible?
  \item Is the program written in terms of the problem domain as much as possible rather than in terms of computer-science or programming-language structures?
\end{todolist}

%%%
\section*{CHECKLIST:\\Good Commenting Technique}
%%%

\subsection*{General}
\begin{todolist}
  \item Can someone pick up the code and immediately start to understand it?
  \item Do comments explain the code's intent or summarize what the code does, rather than just repeating the code?
  \item Is the Pseudocode Programming Process used to reduce commenting time?
  \item Has tricky code been rewritten rather than commented?
  \item Are comments up to date?
  \item Are comments clear and correct?
  \item Does the commenting style allow comments to be easily modified?
\end{todolist}

\subsection*{Statements and Paragraphs}
\begin{todolist}
  \item Does the code avoid endline comments?
  \item Do comments focus on \textit{why} rather than \textit{how}?
  \item Do comments prepare the reader for the code to follow?
  \item Does every comment count? Have redundant, extraneous, and self-indulgent comments been removed or improved?
  \item Are surprises documented?
  \item Have abbreviations beeen avoided?
  \item Is the distinction between major and minor comments clear?
  \item Is code that works around an error or undocumented feature commented?
\end{todolist}

\subsection*{Data Declarations}
\begin{todolist}
  \item Are units on data declarations commented?
  \item Are the ranges of values on numeric data commented?
  \item Are coded meanings commented?
  \item Are limitations on input data commented?
  \item Are flags documented to the bit level?
  \item Has each global variable been commented where it is declared?
  \item Has each global variable been identified as such at each usage, by a naming convention, a comment, or both?
  \item Are magic numbers replaced with named constants or variables rather than just documented?
\end{todolist}

\subsection*{Control Structures}
\begin{todolist}
  \item Is each control statement commented?
  \item Are the ends of long or complex control structures commented, or when possible, simplified so that they don't need comments?
\end{todolist}

\subsection*{Routines}
\begin{todolist}
  \item Is the purpose of each routine commented?
  \item Are other facts about each routine given in comments, when relevant, including input and output data, interface assumptions, limitations, error corrections, global effects, and sources of algorithms?
\end{todolist}

\subsection*{Files, Classes, and Programs}
\begin{todolist}
  \item Does the program have a short document, such as that described in the Book Paradigm, that gives an overall view of how the program is organized?
  \item Is the purpose of each file described?
  \item Are the author's name, e-mail address, and phone numbers in the listing?
\end{todolist}

\end{multicols}
